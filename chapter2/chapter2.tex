\chapter{Building a theoretical model for the dynamics and self-organization of centromeres}

\section{Introduction}

The centromere identity is epigenetically determined by the histone H3 variant CENP-A in organisms with regional centromeres \citep{Warburton1997ImmunolocalizationCentromeres, Vafa1997ChromatinPlate, Earnshaw1985ThreeChromosome, Liu2006MappingCells, Regnier2005CENP-ABubR1, Heun2006, Mendiburo2011, Barnhart2011, Logsdon2015}. Contrary to the epigenetic nature, the chromosomal position of a centromere is stably inherited over generations of cells and only changes if viewed from an evolutionary timescale \citep{Amor2004HumanProgress, Murphy2005DynamicsMaps}. This implies the existence of strategies ensuring the propagation, maintenance and specificity of CENP-A at centromeres. Indeed, various delicate molecular mechanisms conserved across species have been found. 

DNA replication during the S phase inevitably dilutes the number of CENP-A nucleosomes in one chromosome at least by half, where full conservation of old CENP-A is assumed. Therefore, the propagation of CENP-A nucleosomes over generations relies on the deposition of new CENP-A into the chromatin. A conserved positive feedback mechanism has emerged across different species, where old CENP-A is read by proteins that can recruit the CENP-A specific chaperon to deposit new CENP-A \citep{Stirpe2022, McKinley2015}. In vertebrates, CENP-A binds the CCAN protein CENP-C and together they recruit the Mis18 complex, consisting of Mis18$\alpha$, Mis18$\beta$ and Mis18BP1 \citep{Westhorpe2015AMaintenance, Moree2011CENP-CAssembly, Dambacher2012CENP-CChromatin, French2017XenopusAssembly, Wang2014MitoticHJURP, Pan2019MechanismLicensing}. The CENP-A chaperon HJURP bearing CENP-A-H4 heterodimer is then targeted to the centromere by the Mis18 complex \citep{Foltz2009, Dunleavy2009}. CENP-B, the centromeric protein that binds the 17-bp DNA motif CENP-B box, is reported to reinforce this process by stabilising CENP-C \citep{Fachinetti2015, Hoffmann2020, Chardon2022CENP-B-mediatedCentromeres, Masumoto1989ASatellite., Suzuki2004CENP-BLocalization} and is believed to be the 'safety net' for the positive feedback loop \citep{Berg2020}. Similarly, in fission yeast, its Mis18 complex, composed of Mis16, Mis18 and Eic1, and the CENP-A chaperon Scm3 are required for the assembly of its CENP-A ortholog Cnp1 at the centromere \citep{Pidoux2009FissionChromatin, Hayashi2004Mis16Centromeres, Williams2009FissionChromatin}, albeit the requirement for Cnp3, the fission yeast CENP-C, is not conserved \citep{Subramanian2014Eic1Assembly}. Mis18 complex and HJURP orthologs are not found in \textit{Drosophila}. But their functions are combined in one single protein CAL1, which possesses the CENP-A chaperon activity and can self-target to the centromere by interacting with CENP-C and \textit{Drosophila} CENP-A CID \citep{Chen2014, MedinaPritchard2020, Phansalkar2012EvolutionaryDrosophila, Roure2019, Schittenhelm2010}. Additional factors are required for CID deposition in this system, including the FACT complex \citep{Chen2015EstablishmentTranscription} and CAF1 \citep{Furuyama2006Chaperone-mediatedVitro, Boltengagen2016AMelanogaster}. Even in point centromere species budding yeast, whose centromeres are determined genetically, this mechanism is partially conserved. The CBF3 complex recognises the centromeric sequence CDEIII and recruits the chaperon Scm3 to deposit the CENP-A ortholog Cse4 \citep{Camahort2007Scm3Kinetochore, Guan2021StructuralFormation, Cho2011Ndc10Yeast, Mizuguchi2007NonhistoneNucleosomes, Zhou2011StructuralScm3, Meluh1998Cse4pCerevisiae}. Although the exact timing might range from late M to early G1 phase, a common feature of CENP-A deposition shared by various species is that it happens outside the S phase, unlike canonical histone proteins \citep{Fukagawa2014, McKinley2015TheFunction, Stirpe2022}. Consistent with this feature, in vertebrates, the centromeric localisation of HJURP and Mis18BP1 is prevented by CDK1/2-dependent phosphorylation, whose activity is elevated in mitosis and reduced in interphase \citep{Silva2012, Spiller2017MolecularDeposition, Stankovic2017}. The deposition machinery is further regulated by PLK1, which phosphorylates Mis18BP1 and promotes its localisation to the centromere \citep{McKinley2014Polo-likeCentromeres}. Apart from the positive feedback loop and phospho-regulation, proper deposition of CENP-A requires centromeric transcription \citep{Bergmann2012EpigeneticFunction, Catania2015SequenceChromatin, Cardinale2009HierarchicalModifier, Choi2011IdentificationCentromeres, Nakano2008InactivationModifiers, Zhu2018HistoneChromosomes}. It is reasoned that transcription facilitates new CENP-A deposition by evicting embedded non-CENP-A nucleosomes \citep{Chen2015EstablishmentTranscription, Choi2011IdentificationCentromeres, Bobkov2018, Bergmann2012EpigeneticFunction, Bobkov2020, Choi2017TheH3, Prasad2011NewCentromeres}. 

Similar to other epigenetic marks, the maintenance of CENP-A nucleosomes is challenged by chromatin remodelling activities such as replication and transcription. Yet, CENP-A possesses unusual stability that it does not turn over once incorporated into the chromatin \citep{Falk2015, Jansen2007, Bodor2013, Smoak2016Long-TermIdentity}. In replication, this has been attributed to the recycling of disrupted CENP-A-H4 heterodimer by HJURP interacting with MCM2 of the replication fork \citep{Zasadzinska2018, Zasadzinska2013DimerizationDeposition, Huang2015AForks} in a similar manner to the retention of canonical H3 by Asf1$\alpha$ \citep{Richet2015StructuralFork, Clement2015MCM2Fork}. This resulted in the conservative partition of existing CENP-A nucleosomes between the newly replicated sister chromatids \citep{Falk2015, Jansen2007, Bodor2013}. Due to the temporal separation of replicative dilution and new CENP-A deposition, the gaps generated are then filled by another histone H3 variant H3.3 as the placeholder \citep{Dunleavy2011H3.3Phase.}. As mentioned above, transcription can lead to the eviction of incorporated nucleosomes. The general chaperon FACT complex and Spt6 have been proposed to travel with RNA Pol II and reassemble CENP-A nucleosomes dissembled due to transcription \citep{Bobkov2020, Jeronimo2019HistoneModifications, Kato2013Spt6H3, Boltengagen2016AMelanogaster}.

Counter-intuitively, the majority of CENP-A nucleosomes are localised on chromosome arms \citep{Bodor2014}. This could be due to the random deposition of CENP-A caused by the large quantity and the promiscuity to general histone chaperons. Nevertheless, the centromeric enrichment is still over 50-fold higher than arms \citep{Bodor2014}. Apart from the isolation of CENP-A deposition from canonical histone proteins mentioned above, nucleosome eviction by replication, gene expression regulation and PTMs are used to improve the specificity of CENP-A at the centromere. Replication is the main mechanism to remove ectopic CENP-A nucleosomes \citep{Nechemia-Arbely2019, Wang2021PhosphorylationCycle}, with the centromeric ones being proposed to be protected by the CCAN \citep{Nechemia-Arbely2019}. In human and fission yeast cells, CENP-A transcription coincides with their respective deposition timing \citep{Shelby1997AssemblySites, Takahashi2000RequirementYeast, Aristizabal-Corrales2019CellFormation}. Moreover, it has been observed that the expression of CENP-A outside the normal time window would lead to ectopic incorporation in different organisms \citep{Aristizabal-Corrales2019CellFormation, Heun2006, Tomonaga2005CentromereAneuploidy, Au2008AlteredCerevisiae, Olszak2011, McGovern2012CentromereCancer, Athwal2015CENP-ACells, Shrestha2021, Moreno-Moreno2019TheCycle}. Among the various  PTM-based mechanisms for regulating CENP-A localisation, ubiquitin-mediated protein degradation is the most common one \citep{Stirpe2022}. In human \citep{Maehara2010CENP-AMitoses, Lomonte2001DegradationICP0}, \textit{Drosophila} \citep{Moreno-Moreno2019TheCycle, Bade2014TheManner} and budding yeast \citep{Ranjitkar2010AnDomain, Au2013AProteolysis, Mishra2015Pat1Ubiquitination, Zhou2021MolecularPsh1} cells, ubiquitination has been reported to remove CENP-A outside the centromeric regions. 

% low density
%   - the quantitative study
% island pattern
%   - observation 

\nomenclature{CENP-A}{CENtromere Protein A}
\nomenclature{CCAN}{Constitutive Centromere Associated Network}
\nomenclature{HJURP}{Holliday JUnction Recognition Protein}
\nomenclature{CID}{Centromere IDentifier}
\nomenclature{FACT}{Facilitates Chromatin TranscripTion}
 
\section{Methods}
\section{Results}
\section{Discussion}
