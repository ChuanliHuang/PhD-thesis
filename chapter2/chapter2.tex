\chapter{Building a theoretical model for the dynamics and self-organization of centromeres}

\section{Introduction}

The centromere identity is epigenetically determined by the histone H3 variant CENP-A \citep{Warburton1997ImmunolocalizationCentromeres, Vafa1997ChromatinPlate, Earnshaw1985ThreeChromosome, Liu2006MappingCells, Regnier2005CENP-ABubR1, Heun2006, Mendiburo2011, Barnhart2011, Logsdon2015}. Contrary to the epigenetic nature, the chromosomal position of a centromere is stably inherited over generations of cells and only changes if viewed from an evolutionary timescale \citep{Amor2004HumanProgress, Murphy2005DynamicsMaps}. This implies the existence of strategies ensuring the propagation, maintenance and specificity of CENP-A at centromeres. Indeed, various delicate molecular mechanisms conserved across species have been found. 

\nomenclature{CENP-A}{CENtromere Protein A}
 
\section{Methods}
\section{Results}
\section{Discussion}
