\chapter{Investigating the Molecular Mechanisms of Tension-dependent Re-localisation of Shugoshin}

\section{Introduction}
Centromere executes its function of coordinating chromosome segregation partially by controlling the subcellular location of critical regulators. A remarkable example is the tension-dependent re-localisation of shugoshin. 

To be segregated into different daughter cells later, sister chromatids need to be bipolarly attached to the spindle in mitosis or meiosis II, which is called bi-orientation \citep{Tanaka2010Kinetochore-microtubuleBi-orientation}. Bi-oriented sister chromatids come under tension due to the counteraction between the resistance of cohesion and the pulling of microtubules. Cells sense a lack of tension to destabilise erroneous microtubule-kinetochore attachment and delay cell cycle progression to allow time for the repair, a process termed error correction \citep{Nicklas1997HowChromosomes, Nicklas1994, Tanaka2010Kinetochore-microtubuleBi-orientation}. The conserved shugoshin protein family is important for the establishment of tension \citep{Watanabe2005, Clift2011, Gutierrez-Caballero2012Shugoshins:Centromere, Marston2015, Zhang2020FunctioningMitosis}. It is named 'guardian spirit' in Japanese because of its canonical function of cohesion protection in meiosis I, which is well conserved across eukaryotes \citep{Lister2010Age-relatedSgo2, Llano2008Shugoshin-2Mice, Lee2008, Rattani2013Sgol2Oocytes, Marston2004a, Kitajima2004a, Katis2004, Rabitsch2004TwoII, Kerrebrock1992TheDifferentiation, Cromer2013CentromericInterkinesis, Zamariola2014SHUGOSHINsThaliana, Wang2011OsSGO1Meiosis, Hamant2005AFunctions}. In vertebrates, shugoshin additionally protects centromeric cohesion from Wapl-mediated cohesin removal pathway in mitosis \citep{Rivera2009ShugoshinExtracts, Shintomi2009ReleasingSgo1, Huang2007, Tang2006a, McGuinness2005ShugoshinCells, Kitajima2005, Salic2004VertebrateMitosis}. Despite the slight differences in molecular mechanisms, both of them are carried out by the effector protein PP2A-B56 \citep{Xu2009StructureInteraction, Ueki2021AMitosis}. Shugoshin further assists cohesion protection by directly sequestering separase with checkpoint component Mad2\citep{Rattani2013Sgol2Oocytes, Hellmuth2020Securin-independentShugoshinMAD2, Orth2011ShugoshinMad2}. Apart from cohesion protection, shugoshin also promotes bi-orientation in two ways. It facilitates error correction through supporting CPC localisation at the centromere in various species \citep{Peplowska2014, Nerusheva2014, Verzijlbergen2014, Tsukahara2010a, Yamagishi2010, Hadders2020UntanglingMitosis, Broad2020AuroraCells, Kawashima2007, Vanoosthuyse2007, Rivera2012}. In budding yeast, shugoshin interacts with condensin, contributing to the geometry of the centromeric region biased towards a shape where sister kinetochores are more easily captured by microtubules from the opposite poles \citep{Indjeian2007, Haase2012Bub1Dynamics, Verzijlbergen2014, Peplowska2014, Yahya2020}. 

 Interestingly, the location of shugoshin has been found to change in response to tension across species. Human Sgo1 and Sgo2 redistribute from the inner centromere towards the kinetochore upon tension establishment in mitosis \citep{Huang2007, Lee2008, Liu2013, Asai2020}. Mouse Sgo2 follows the same pattern in meiosis II \citep{Lee2008, Gomez2007}. In budding yeast, tension removes its only shugoshin protein Sgo1 from peri-centromere \citep{Eshleman2014, Nerusheva2014, Paldi2020ConvergentPericentromeres}. Although the effect of tension has not been directly studied, locations are different between metaphase and anaphase for \textit{Drosophila} MEI-S332 (mitosis and meiosis II) and fission yeast Sgo2 (mitosis) \citep{Clarke2005, Kawashima2007}, raising the possibility that they also undergo tension-dependent re-localisation. 

Re-localisation of shugoshin is proposed to be key to tension sensing \citep{Marston2015}. Screening in budding yeast identified Sgo1 being required to delay anaphase onset in the absence of tension \citep{Indjeian2005a}. Later, it was shown to be due to its role in preventing SAC silencing \citep{Jin2013TheAttachment}. Given the function of Sgo1 to support CPC localisation, it was reasoned that tension-dependent re-localisation of Sgo1 triggers the removal of CPC from the centromere, leading to SAC silencing \citep{Nerusheva2014}. In support of this model, artificially tethering Sgo1 to the kinetochore caused prolonged metaphase \citep{Su2021SumoylationAnaphase}. In mammals, the re-localisation is thought to inactivate cohesin protection \citep{Lee2008}. Indeed, impaired human Sgo1 re-localisation increased lagging chromosomes in anaphase \citep{Liu2013}. Besides, at least in budding yeast, the re-localisation of shugoshin might be the signal for chromosome condensation. The key condensation player condensin spreads from the centromere to chromosome arms in a tension-dependent manner \citep{Leonard2015}. As its interactor with a similar response to tension, Sgo1 is likely to mediate the spread. Consistent with this idea, mitotic chromosome condensation is abolished in \textit{sgo1} mutant \citep{Kruitwagen2018}. 

The mechanism of shugoshin re-localisation is less well understood. 


\nomenclature{SAC}{Spindle Assembly Checkpoint}
\nomenclature{CPC}{Chromosome Passenger Complex}
\nomenclature{APC}{Anaphase promoting complex}
\nomenclature{PP2A}{Protein Phosphatase 2A}

\section{Results}
\subsection{}
\section{Discussion}
