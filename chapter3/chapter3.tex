\chapter{Investigating the Molecular Mechanisms of Tension-dependent Re-localisation of Shugoshin}

\section{Introduction}
Centromere executes its function of coordinating chromosome segregation partially, if not mainly, through controlling subcellular location of critical regulators. A remarkable example is tension-dependent re-localisation of shugoshin. 

Conserved shugoshin protein family is a versatile peri-centromeric adaptor that engages various effector proteins important for accurate chromosome segregation in both mitosis and meiosis \citep{Watanabe2005, Clift2011, Gutierrez-Caballero2012Shugoshins:Centromere, Marston2015, Zhang2020FunctioningMitosis}.

To be segregated into different daughter cells later, sister chromatids need to be bipolarly attached to the spindle during metaphase in mitosis or metaphase II in meiosis, which is termed as bi-orientation \citep{Tanaka2010Kinetochore-microtubuleBi-orientation}. Bi-oriented sister chromatids come under 
tension due to the counteraction between resistance of cohesion and pulling of microtubules. Cells sense lack of tension to identify erroneous microtubule-kinetochore attachment and delay cell cycle progression \citep{Nicklas1997HowChromosomes, Nicklas1994}. Location of shugoshin has been found to change in response to tension across species. Human Sgo1 and Sgo2 redistribute from inner centromere towards kinetochore upon tension establishment in mitosis \citep{Huang2007, Lee2008, Liu2013, Asai2020}. Mouse Sgo2 follows the same pattern in meiosis II \citep{Lee2008, Gomez2007}. In budding yeast, tension triggers removal of its only shugoshin protein Sgo1 from peri-centromere \citep{, Eshleman2014, Nerusheva2014, Paldi2020ConvergentPericentromeres}. Although effect of tension has not been directly studied, locations are different between metaphase and anaphase for \textit{Drosophila} MEI-S332 (mitosis and meiosis II) and fission yeast Sgo2 (mitosis) \citep{Clarke2005, Kawashima2007}, raising the possibility that they also undergo tension-dependent re-localisation. 

\section{Results}
\subsection{}
\section{Discussion}
