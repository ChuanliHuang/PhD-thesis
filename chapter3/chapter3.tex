\chapter{Investigating the Molecular Mechanisms of Tension-dependent Re-localisation of Shugoshin}

\section{Introduction}
Centromere executes its function of coordinating chromosome segregation partially by controlling the subcellular location of critical regulators. A remarkable example is the tension-dependent re-localisation of shugoshin. 

To be segregated into different daughter cells later, sister chromatids need to be bipolarly attached to the spindle in mitosis or meiosis II, which is called bi-orientation \citep{Tanaka2010Kinetochore-microtubuleBi-orientation}. Bi-oriented sister chromatids come under tension due to the counteraction between the resistance of cohesion and the pulling of microtubules. Cells sense a lack of tension and destabilise erroneous microtubule-kinetochore attachment, a process termed error correction, meanwhile delaying cell cycle progression until tension is generated \citep{Nicklas1997HowChromosomes, Nicklas1994, Tanaka2010Kinetochore-microtubuleBi-orientation}. The conserved shugoshin protein family is important for the establishment of tension \citep{Watanabe2005, Clift2011, Gutierrez-Caballero2012Shugoshins:Centromere, Marston2015, Zhang2020FunctioningMitosis}. The name itself, 'guardian spirit' in Japanese, comes from its well-conserved canonical function of cohesion protection in meiosis I \citep{Lister2010Age-relatedSgo2, Llano2008Shugoshin-2Mice, Lee2008, Rattani2013Sgol2Oocytes, Marston2004a, Kitajima2004a, Katis2004, Rabitsch2004TwoII, Kerrebrock1992TheDifferentiation, Cromer2013CentromericInterkinesis, Zamariola2014SHUGOSHINsThaliana, Wang2011OsSGO1Meiosis, Hamant2005AFunctions, Ma2021MeikinI, Miyazaki2017HierarchicalI}. In vertebrates, shugoshin also protects centromeric cohesion from Wapl-mediated cohesin removal pathway in mitosis \citep{Rivera2009ShugoshinExtracts, Shintomi2009ReleasingSgo1, Huang2007, Tang2006a, McGuinness2005ShugoshinCells, Kitajima2005, Salic2004VertebrateMitosis, Liang2019ACells}. Shugoshin further assists cohesion protection by directly sequestering separase with the SAC component Mad2\citep{Rattani2013Sgol2Oocytes, Hellmuth2020Securin-independentShugoshinMAD2, Orth2011ShugoshinMad2}. Apart from cohesion protection, shugoshin additionally promotes bi-orientation by facilitating error correction \citep{Meppelink2015Shugoshin-1Bi-orientation, Huang2007, Peplowska2014, Nerusheva2014, Verzijlbergen2014, Tsukahara2010a, Yamagishi2010, Hadders2020UntanglingMitosis, Broad2020AuroraCells, Kawashima2007, Vanoosthuyse2007, Rivera2012} and, at least in yeast, contributing to the geometry of the centromeric region biased towards a shape where sister kinetochores are more easily captured by microtubules from the opposite poles \citep{Indjeian2007, Haase2012Bub1Dynamics, Verzijlbergen2014, Peplowska2014, Sane2021ShugoshinDisassembly}. Despite different molecular details in various species, all these functions are implemented by shugoshin acting as an adaptor recruiting different effector proteins to the centromeric region, including PP2A-B56 \citep{Xu2009StructureInteraction, Ueki2021AMitosis}, CPC \citep{Abad2022MechanisticCPC}, MCAK \citep{Tanno2010} and condensin \citep{Verzijlbergen2014, Yahya2020}. 

Interestingly, the location of shugoshin has been found to change in response to tension across species. Human Sgo1 and Sgo2 redistribute from the inner centromere towards the kinetochore upon tension establishment in mitosis \citep{Huang2007, Lee2008, Liu2013, Asai2020}. Mouse Sgo2 follows the same pattern in meiosis II \citep{Lee2008, Gomez2007}. In budding yeast, tension removes its only shugoshin protein Sgo1 from peri-centromere \citep{Eshleman2014, Nerusheva2014, Paldi2020ConvergentPericentromeres}. Although the effect of tension has not been directly studied, locations are different between metaphase and anaphase for \textit{Drosophila} MEI-S332 (mitosis and meiosis II) and fission yeast Sgo2 (mitosis) \citep{Clarke2005, Kawashima2007}, raising the possibility that they also undergo tension-dependent re-localisation. 

Re-localisation of shugoshin is functionally relevant. It has been proposed as key to tension sensing \citep{Marston2015}. Screening in budding yeast identified Sgo1 being required to delay anaphase onset in the absence of tension \citep{Indjeian2005a}. Later, it was shown to be due to its role in preventing SAC silencing \citep{Jin2013TheAttachment}. Given the function of Sgo1 to support CPC localisation, it was reasoned that tension-dependent re-localisation of Sgo1 triggers the removal of CPC from the centromere, leading to SAC silencing \citep{Nerusheva2014}. In support of this model, artificially tethering Sgo1 to the kinetochore caused prolonged metaphase \citep{Su2021SumoylationAnaphase}. In mammals, the re-localisation is thought to inactivate cohesin protection \citep{Lee2008}. Indeed, impaired human Sgo1 re-localisation increased lagging chromosomes in anaphase \citep{Liu2013}. Besides, at least in budding yeast, the re-localisation of shugoshin might be the signal for chromosome condensation. The key condensation player condensin spreads from the centromere to chromosome arms in a tension-dependent manner \citep{Leonard2015}. As its interactor with a similar response to tension, Sgo1 potentially mediates the spread. Consistent with this idea, mitotic chromosome condensation is abolished in \textit{sgo1} mutant \citep{Kruitwagen2018}. 

It is well established in different organisms that the SAC component Bub1 kinase at the kinetochore phosphorylates threonine 120 of histone H2A (serine 121 in budding yeast) around centromeric chromatin, providing a high-affinity marker for shugoshin. Nevertheless, the localisation of shugoshin is further regulated by complicated, not fully understood, reversible PTMs from the interplay of various factors. In human cells, a step-wise recruitment model of Sgo1 has been proposed. Sgo1 is first recruited proximal to the kinetochore due to direct binding to nucleosomes with H2A-pT120. Pol II-mediated centromeric transcription then disengages Sgo1 from nucleosomes, allowing it to reach the inner centromere. Finally, Sgo1 phosphorylated at threonine 346 by CDK is captured by cohesin there. A number of other factors supporting shugoshin localisation have been reported in different species, including HP1, H3, CENP-A, CPC and PP2A, while Polo kinase, SET, KAT2A (Gcn5 in budding yeast) and PP2A were suggested to de-localise it. 

The mechanism of shugoshin re-localisation is less well understood. 


\nomenclature{SAC}{Spindle Assembly Checkpoint}
\nomenclature{CPC}{Chromosome Passenger Complex}
\nomenclature{PP2A}{Protein Phosphatase 2A}
\nomenclature{MCAK}{Mitotic Centromere-Associated Kinesin}
\nomenclature{SET}{SET nuclear proto-oncogene}
\nomenclature{CDK}{Cyclin-Dependent Kinase}
\nomenclature{Pol II}{RNA POLymerase II}
\nomenclature{HP1}{Heterochromatin Protein 1}
\nomenclature{KAT2A}{lysine AcetylTransferase 2A}
\nomenclature{PTM}{Post Translational Modification}


\section{Results}
\subsection{}
\section{Discussion}
