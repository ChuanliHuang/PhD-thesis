\chapter{Introduction}
\section{Centromere is a specialised locus for chromosome segregation}

 The growth and reproduction of all living organisms rely on cell division. Delivering an intact genome to the daughter cell challenges every dividing cell. Eukaryotes package their genomes into chromosomes, which must be replicated and segregated faithfully during each cell division. Severe diseases, such as cancer, genetic disorder and miscarriage, could arise if chromosome segregation is compromised \citep{Jallepalli2001ChromosomeMystery, Draviam2004ChromosomeStability, Wasielak-Politowska2022ChromosomeAging, Losada2014CohesinBeyond}. 
 
 Centromere is the specialised locus for chromosome segregation, where the chromosome is attached to and pulled by the spindle \citep{Westhorpe2015AMaintenance, McKinley2015TheFunction, Talbert2020WhatCentromere, Fukagawa2014}. First described by \cite{Flemming1882ZellsubstanzZelltheilung} as primary constriction, it is visualised as the intersection point of the iconic X-shaped mitotic chromosome under a microscope. Lack of centromere, or being acentric, leads to chromosome loss during cell division. Although the appreciation of centromere function has been established since its discovery, the molecular feature underneath remained elusive due to the species-to-species variations and the complexity of players involved. \cite{Darlington1936TheEnquiry} therefore recommended defining the centromere in terms of the function rather than the form. Nevertheless, advances in molecular biology shed light on the form of centromere. In most species, the centromeric DNA sequence plays an important but not definitive role in centromere specification \citep{Hoffmann2020, Harrington1997FormationMicrochromosomes, Catania2015SequenceChromatin, Iwata-Otsubo2017, Kasinathan2018Non-B-FormCentromeres, Shukla2018CentromereCycle, Logsdon2019, Murillo-Pineda2020}. It is instead determined epigenetically by the histone H3 variant CENP-A \citep{Warburton1997ImmunolocalizationCentromeres, Vafa1997ChromatinPlate, Earnshaw1985ThreeChromosome, Liu2006MappingCells, Regnier2005CENP-ABubR1, Heun2006, Mendiburo2011, Barnhart2011, Logsdon2015}. The macromolecular protein complex consisting of around 100 subunits, kinetochore, is formed on the centromere to mediate the physical connection between the chromosome and spindle microtubules and act as a signalling hub to monitor the interaction \citep{Musacchio2017AFunction, McAinsh2022TheKinetochores, Cheeseman2014TheKinetochore, Hara2018KinetochoreExit}. Apart from the progress in the form, the mechanisms centromere executing the function have also been elucidated over the years \citep{Tanaka2013, Zhou2020EmergentChromosomes}. In the following sections, the form and function of the centromere will be described in more detail. 

\section{The form of centromere}
\subsection{The genetics aspects}

The structure of centromeric DNA and its functional importance varies dramatically between species. Based on the chromosomal distribution, centromeres are classified as monocentromere, where the centromere is localised at a restricted region of the chromosome, and holocentromere, where the centromere spreads the entire length of the chromosome. Depending on size, the monocentromere can be further divided into point centromere, which contains a short DNA sequence of just over 100 bp, and regional centromere, which could be up to megabases long. The importance of underlying DNA sequences to centromere function could also be very different, ranging from pure genetic to mainly epigenetic. 

Point centromere, with a notable example budding yeast \textit{Saccharomyces cerevisiae}, is the simplest form and the first investigated at the molecular level. A budding yeast centromere consists of three elements \citep{Carbon1984StructuralCEN3}: CDEI, an 8-bp sequence that binds Cfb1 for H3 nucleosome eviction \citep{Niedenthal1993Cpf1I, Henikoff2011EpigenomeResolution}; CDEII, an AT-rich, around 80-bp sequence accommodating a single centromeric nucleosome containing Cse4, the budding yeast CENP-A, for kinetochore assembly \citep{Furuyama2007CentromereYeast, Henikoff2014TheVivo, Krassovsky2012TripartiteYeast} and CDEIII, a 25-bp sequence bound by the CBF3 complex \citep{Yan2018ArchitectureKinetochore}, which recruits the Cse4 chaperon Scm3, the budding yeast HJURP, for Cse4-containing nucleosome deposition \citep{Stoler2007Scm3Localization, Camahort2007Scm3Kinetochore}. Centromere specification is strictly genetic in this organism \citep{Clarke1980IsolationChromosomes, J1986SingleCerevisiae, Kingsbury1991Centromere-dependentVitro}, which could be explained by its unique centromere biology that factors for Cse4 nucleosome deposition are recruited by specific DNA sequences. This is in line with the observation that the exact sequences of CDEI and CDEIII are conserved across all 16 centromeres of budding yeast \citep{Clarke1998Centromeres:Eukaryotes, Baker2005GeneticCerevisiae}. As for the centromeric nucleosome accommodating sequence CDEII, only the length and AT abundance are conserved, supporting the notion from regional centromeres that CENP-A nucleosome binding does not require specific DNA sequences \citep{Bensasson2011EvidenceCentromeres}. The phylogeny indicated that point centromere species evolved from regional centromere species and this transition coincided with the emergence of 2-micron plasmid, a multicopy circular DNA capable of self-propagating in budding yeast \citep{Malik2009MajorComplexity}. Intriguingly, the 2-micron plasmid also uses a single locus called \textit{STB} to assemble a partitioning complex, which includes components of segregation machinery for normal chromosomes such as Cse4 and cohesin, for its association with the spindle microtubule \citep{Rizvi2018TheCerevisiae, Huang2011Evolution, Ghosh2007FaithfulSisters}. The outstanding resemblance led to the hypothesis that the point centromere is the domestication of the self-propagating locus of parasitic plasmids \citep{Malik2009MajorComplexity}. 

The regional centromere is the most common type of centromere. A typical regional centromere is AT-rich and possesses a modular structure composed of a CENP-A-nucleosome-accommodating central core flanked by heterochromatic domains called peri-centromere. The central core usually contains satellite DNA, short sequences repeated a large number of times, whereas the peri-centromere has less patterned sequences \citep{Talbert2020WhatCentromere, McKinley2015TheFunction, Wong2020LessonsChromosomes, Muller2019TheArchitecture}. As one of the fastest-evolving loci across the genome, the precise sequence of the centromere is extremely diverse among species \citep{Melters2013ComparativeEvolution}. Notably, due to the incompatibility of second-generation sequencing and repetitive sequence, deciphering the centromeric DNA sequence has been challenging. In fission yeast \textit{Schizosaccharomyces pombe}, the central core consists of non-repetitive \textit{cc} and inverted repeats \textit{imr} while the peri-centromere possess less ordered \textit{otr} composed of \textit{dg} and \textit{dh} repeats \citep{Chikashige1989CompositeSites, Clarke1993StructureCentromeres, Murakami1991StructureRegion, Nakaseko1986ChromosomeYeast, Nakaseko1987AChromosomes., Steiner1993CentromeresLoci}. Fruit fly \textit{Drosophila melanogaster} has a central core made up of a retroelement-enriched island flanked with AATAT and AAGAG satellites, and peri-centromeric chromatin containing mixed different types of short satellites that are neither conserved among chromosomes nor specific to the centromere \citep{Talbert2018SimpleSpecies, Wong2020LessonsChromosomes, Chang2019IslandsCentromeres}. House mouse \textit{Mus musculus} centromeres are close to telomeres, which is termed acrocentric. The central core and the peri-centromere consist of 120-bp minor satellites and 234-bp major satellites, respectively \citep{Komissarov2011TandemlyGenome, Kuznetsova2006High-resolutionDNA}. Primate including human \textit{Homo sapiens} central core contains 171-bp monomers, named $\alpha$-satellite, arranged into HOR arrays of different lengths, whereas the flanking peri-centromere is built with less structured monomeric $\alpha$-satellites that are less recognisable \citep{Maio1971DNAAethiops, Rosenberg1978HighlySIMIANSIMIANSIMIANSIMIANSIMIAN, Manuelidis1978ComplexDNAs, Manuelidis1978ChromosomalDNAs, Aldrup-MacDonald2014TheGenomics, Logsdon2021The8}. Unlike the point centromere, the centromeric DNA sequence is neither sufficient nor necessary for the function of the regional centromere \citep{McKinley2015TheFunction}. The former was indicated by the observations that the dicentric chromosomes due to the fusion of two normal chromosomes only assembled centromeric proteins, such as CENP-A, at one of the two endogenous centromeres \citep{Earnshaw1985ThreeChromosome, Steiner1994AYeast, Higgins2005EngineeredPlasticity, Sato2012EpigeneticChromosomes, Sullivan1995IdentificationCentromeres, Lange2009IsodicentricPalindromes}. The latter was evidenced by the fact the ectopic centromere, neocentromere, can form on chromosomal regions whose sequences bear little similarity with the canonical centromeres \citep{Marshall2008Neocentromeres:Evolution, Voullaire1993ACentromere, Tyler-Smith1999TransmissionGenerations, Amor2004HumanProgress}. The epigenetic notion was later confirmed by the elucidation of the requirement of CENP-A for centromere function and localisation \citep{Warburton1997ImmunolocalizationCentromeres, Vafa1997ChromatinPlate, Liu2006MappingCells, Regnier2005CENP-ABubR1, Heun2006, Mendiburo2011, Barnhart2011, Logsdon2015, Logsdon2019}. However, experimental results that cloned regional centromeric sequences were sufficient to support the inheritance of exogenous DNA suggest a contribution of sequence to \textit{de novo} centromere formation \citep{Hahnenberger1989ConstructionPombe., Haaf1992IntegrationSegregation, Harrington1997FormationMicrochromosomes, Ikeno1998ConstructionChromosomes}. This could partially be explained by the presence of the sequence-specific DNA-binding centromeric protein CENP-B \citep{Masumoto1989ASatellite., Muro1992CentromereBox., Earnshaw1985IdentificationScleroderma}. CENP-B is not essential for the centromere function as evidenced by that CENP-B knockout mice are still viable \citep{Kapoor1998TheMice, Perez-Castro1998CentromericAbnormalities, Hudson1998CentromereWeights}. But the centromere of the Y chromosome, which lacks CENP-B binding sequences called CENP-B box, failed to generate HACs \textit{in vivo} \citep{Harrington1997FormationMicrochromosomes, Grimes2002-SatelliteFormation}, consistent with the idea that centromeric DNA sequences facilitate the establishment of a functional centromere. The molecular mechanism was later uncovered that CENP-B interacts with both CENP-A and the CCAN component CENP-C to facilitate CENP-A deposition and kinetochore assembly \citep{Chardon2022CENP-B-mediatedCentromeres, Fachinetti2013, Fachinetti2015, Fujita2015StableNucleosome}. As mentioned above, the repetitiveness of centromeric sequences is a conserved feature in various regional centromere species and therefore evolutionarily preferred. Moreover, newly positioned centromeres from speciation, the ENCs, tend to gradually acquire repetitive sequences over time \citep{Rocchi2011CentromereMammals, Kasai2003ChromosomeEvolution}. The hypothesised mode of action is that younger sequences were inserted at the central core, pushing the old sequences towards the flank \citep{Locke2011ComparativeGenomes, Piras2010UncouplingEquus, Ventura2001CentromereEvolution, Kalitsis2012TheCentromere}, which was supported by the recently revealed complete human centromere sequences \citep{Logsdon2021The8}. The evolutionary preference for repetitive sequences also implied its importance to centromere function. It is speculated that tandem repeats might prevent the CENP-A domains from sliding along the centromere over cell cycles \citep{Nergadze2018BirthDomains}. 

Holocentromere

\nomenclature{CDE}{Centomere DNA Element}
\nomenclature{CBF}{Centromeric DNA Binding Factor}
\nomenclature{ENC}{Evolutionarily New Centromere}
\nomenclature{HOR}{Higher-Order Repeat}
\nomenclature{HAC}{Human Artificial Chromosome}

\subsection{The epigenetics aspects}
\subsection{The kinetochore}

Nomenclature confusion

\section{The function of centromere}
\subsection{Error-free kinetochore-microtubule attachment}
\subsection{Robust peri-centromeric cohesion}
\subsection{Advanced replication timing}
\subsection{Chromosome condensation initialisation}
\section{Aims of this study}
\subsection{Building a theoretical model for centromere spatial organisation and propagation}
\subsection{Investigating the molecular mechanisms of tension-dependent re-localisation of shugoshin}
