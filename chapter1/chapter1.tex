\chapter{Introduction}
\section{Centromere is a specialised locus for chromosome segregation}

 The growth and reproduction of all living organisms rely on cell division. Delivering an intact genome to the daughter cell challenges every dividing cell. Eukaryotes package their genomes into chromosomes, which must be replicated and segregated faithfully during each cell division. Severe diseases, such as cancer and miscarriage, could arise if chromosome segregation is compromised. 
 
 Centromere is the specialised locus for chromosome segregation, where the chromosome is attached to and pulled by the spindle \citep{Westhorpe2015AMaintenance, McKinley2015TheFunction, Talbert2020WhatCentromere, Fukagawa2014}. First described by \cite{Flemming1882ZellsubstanzZelltheilung} as primary constriction, it is visualised as the intersection point of the iconic X-shaped mitotic chromosome under a microscope. Lack of centromere, or being acentric, leads to chromosome loss during cell division. Although the appreciation of centromere function has been established since its discovery, the molecular feature underneath remained elusive due to the species-to-species variations and the complexity of players involved. \cite{Darlington1936TheEnquiry} therefore recommended defining the centromere in terms of the function rather than the form. Nevertheless, advances in molecular biology shed light on the form of centromere. In most species, the centromeric DNA sequence plays an important but not definitive role in centromere specification \citep{Hoffmann2020, Harrington1997FormationMicrochromosomes, Catania2015SequenceChromatin, Iwata-Otsubo2017, Kasinathan2018Non-B-FormCentromeres, Shukla2018CentromereCycle, Logsdon2019, Murillo-Pineda2020}. It is instead determined epigenetically by the histone H3 variant CENP-A \citep{Warburton1997ImmunolocalizationCentromeres, Vafa1997ChromatinPlate, Earnshaw1985ThreeChromosome, Liu2006MappingCells, Regnier2005CENP-ABubR1, Heun2006, Mendiburo2011, Barnhart2011, Logsdon2015}. The macromolecular protein complex consisting of around 100 subunits, kinetochore, is formed on the centromere to mediate the physical connection between the chromosome and spindle microtubules and act as a signalling hub to monitor the interaction \citep{Musacchio2017AFunction, McAinsh2022TheKinetochores, Cheeseman2014TheKinetochore, Hara2018KinetochoreExit}. Apart from the progress in the form, the mechanisms centromere executing the function have also been elucidated over the years \citep{Tanaka2013, Zhou2020EmergentChromosomes}. In the following sections, the form and function of the centromere will be described in more detail. 

\section{The form of centromere}
\subsection{The genetics aspects}

The structure of centromeric DNA and its functional importance varies dramatically between species. Based on the chromosomal distribution, centromeres are classified as monocentromere, where the centromere is localised at a restricted region of the chromosome, and holocentromere, where the centromere spreads the entire length of the chromosome. Depending on size, the monocentromere can be further divided into point centromere, which contains a short DNA sequence of just over 100 bp, and regional centromere, which could be up to megabases long, as exemplified by the human centromere. The importance of underlying DNA sequences to centromere function could also be very different, ranging from pure genetic to mainly epigenetic. 

Point centromere, with a notable example budding yeast \textit{Saccharomyces cerevisiae}, is the simplest form and the first investigated at the molecular level. A budding yeast centromere consists of three elements \citep{Carbon1984StructuralCEN3}: CDEI, an 8-bp sequence that binds Cfb1 for H3 nucleosome eviction \citep{Niedenthal1993Cpf1I, Henikoff2011EpigenomeResolution}; CDEII, an AT-rich, around 80-bp sequence accommodating a single centromeric nucleosome containing Cse4, the budding yeast CENP-A, for kinetochore assembly \citep{Furuyama2007CentromereYeast, Henikoff2014TheVivo, Krassovsky2012TripartiteYeast} and CDEIII, a 25-bp sequence bound by the CBF3 complex \citep{Yan2018ArchitectureKinetochore}, which recruits the Cse4 chaperon Scm3, the budding yeast HJURP, for Cse4-containing nucleosome deposition \citep{Stoler2007Scm3Localization, Camahort2007Scm3Kinetochore}. Centromere specification is strictly genetic in this organism \citep{Clarke1980IsolationChromosomes, J1986SingleCerevisiae, Kingsbury1991Centromere-dependentVitro}, which could be explained by its unique centromere biology that factors for Cse4 nucleosome deposition are recruited by specific DNA sequences. This is in line with the observation that the exact sequences of CDEI and CDEIII are conserved across all 16 centromeres of budding yeast \citep{Clarke1998Centromeres:Eukaryotes, Baker2005GeneticCerevisiae}. As for the centromeric nucleosome accommodating sequence CDEII, only the length and AT abundance are conserved, supporting the notion from regional centromeres that CENP-A nucleosome binding does not require specific DNA sequences \citep{Bensasson2011EvidenceCentromeres}. The phylogeny indicated that point centromere species evolved from regional centromere species and this transition coincided with the emergence of 2-micron plasmid, a multicopy circular DNA capable of self-propagating in budding yeast \citep{Malik2009MajorComplexity}. Intriguingly, the 2-micron plasmid also uses a single locus called \textit{STB} to assemble a partitioning complex, which includes Cse4 and cohesin, for its association with the spindle microtubule \citep{Rizvi2018TheCerevisiae, Huang2011Evolution, Ghosh2007FaithfulSisters}. The outstanding resemblance led to the hypothesis that the point centromere is the domestication of the self-propagating locus of parasitic plasmids \citep{Malik2009MajorComplexity}. 

Regional centromere

common feature
diversity
sequence importance
evolution (centromere drive)

\nomenclature{CDE}{Centomere DNA Element}
\nomenclature{CBF}{Centromeric DNA Binding Factor}

\subsection{The epigenetics aspects}
\subsection{The kinetochore}

Nomenclature confusion

\section{The function of centromere}
\subsection{Error-free kinetochore-microtubule attachment}
\subsection{Robust peri-centromeric cohesion}
\subsection{Advanced replication timing}
\subsection{Initialising chromosome condensation}
\section{Aims of this study}
\subsection{Building a theoretical model for centromere spatial organisation and propagation}
\subsection{Investigating the molecular mechanisms of tension-dependent re-localisation of shugoshin}
