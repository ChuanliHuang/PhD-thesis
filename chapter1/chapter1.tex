\chapter{Introduction}
\section{Centromere is a specialised locus for chromosome segregation}

 The growth and reproduction of all living organisms rely on cell division. A challenge dividing cells face is delivering an intact genome to the daughter cell. Eukaryotes package their genomes into chromosomes, which must be replicated and segregated faithfully during each cell division. Severe diseases, such as cancer and miscarriage, could arise if chromosome segregation is compromised. 
 
 Centromere is the specialised locus for chromosome segregation, where the chromosome is attached to and pulled by the spindle \citep{Westhorpe2015AMaintenance, McKinley2015TheFunction, Talbert2020WhatCentromere, Fukagawa2014}. First described by \cite{Flemming1882ZellsubstanzZelltheilung} as primary constriction, it is visualised as the intersection point of the iconic X shape mitotic chromosome under a microscope. Lack of centromere, or being acentric, leads to the loss of the chromosome during cell division. Although the appreciation of centromere function has been established since its discovery, the molecular feature underneath remained elusive due to the species-to-species variations and the complexity of players involved. \cite{Darlington1936TheEnquiry} therefore recommended defining the centromere in terms of the function rather than the form. Nevertheless, advances in molecular biology shed light on the form of centromere. In most species, the centromeric DNA sequence plays an important but not definitive role in centromere specification \citep{Hoffmann2020, Harrington1997FormationMicrochromosomes, Catania2015SequenceChromatin, Iwata-Otsubo2017, Kasinathan2018Non-B-FormCentromeres, Shukla2018CentromereCycle, Logsdon2019, Murillo-Pineda2020}. It is instead determined epigenetically by the histone H3 variant CENP-A \citep{Warburton1997ImmunolocalizationCentromeres, Vafa1997ChromatinPlate, Earnshaw1985ThreeChromosome, Liu2006MappingCells, Regnier2005CENP-ABubR1, Heun2006, Mendiburo2011, Barnhart2011, Logsdon2015}. The macromolecular protein complex consisting of around 100 subunits, kinetochore, is formed on the centromere to mediate the physical connection between the chromosome and spindle microtubules and act as a signalling hub to monitor the interaction \citep{Musacchio2017AFunction, McAinsh2022TheKinetochores, Cheeseman2014TheKinetochore, Hara2018KinetochoreExit}. Apart from the progress in the form, the mechanisms centromere executing the function have also been elucidated over the years. In the following sections, the form and function of the centromere will be described in more detail. 

\section{The form of centromere}
\subsection{The genetics aspects}
\subsection{The epigenetics aspects}
\subsection{The kinetochore}

Nomenclature confusion

\section{The function of centromere}
\subsection{Error-free kinetochore-microtubule attachment}
\subsection{Robust peri-centromeric cohesion}
\subsection{Advanced replication timing}
\subsection{Initialising chromosome condensation}
\section{Aims of this study}
\subsection{Building a theoretical model for centromere spatial organisation and propagation}
\subsection{Investigating the molecular mechanisms of tension-dependent re-localisation of shugoshin}