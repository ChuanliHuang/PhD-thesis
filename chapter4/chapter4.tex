\chapter{Discussion}
\section{Final discussion}

% Benefits of Mathematical Modeling
% Humans make models to make sense of the world—to organize information and provide predictability. In the biological sciences, models are commonly expressed using words or diagrams, which often suffer from a lack of preci- sion. Mathematics is the most precise language we have for describing the processes we think are happening and for comparing this with what we can observe. Even attempting to make a mathematical model helps to clarify ideas, define assumptions and identify missing information. A working mathemati- cal description is a potent tool, simplifying large amounts of information, providing precise predictions to aid further experimental tests, and allowing rapid exploration of the range of properties of the system to help understand design–function relationships. Importantly, a mathematical model also aids generalization of insights gained from one system to other systems.
% However, the rigour of a mathematical model is no guarantee that it is correct. Models will always be based on incomplete and often incorrect knowledge. They should be judged by their usefulness—their ability to explain current knowledge and to guide the attainment of new knowledge.
% Our Modeling Approach—Simplified Stochastic Simulations
% There are many kinds of mathematical models (many examples are provided in this book). Our approach has been to use computer-based stochastic sim- ulations, which are a relatively “mathematics-light” way to add or remove possible processes and which naturally provide noise. The battle between a parameter-averse physicist (“We don’t need to put that in”) and a parameter- addicted biologist (“But we know it’s not that simple”) results in a compro- mise where complex processes are represented in simplified or idealized ways. Our hope is that this simplification exposes rather than obscures the funda- mentals of the system. Details are important, of course, but fundamentals are more important.