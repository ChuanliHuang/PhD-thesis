\chapter{Discussion}

\section{Final discussion}

Accurate chromosome segregation is a fundamental requirement for eukaryotic life, which relies on the specialised locus called the centromere. The spatial regulation of critical proteins is the key to the realisation of centromeric function. This work aimed to study the spatial organisation and dynamics of two vital centromeric proteins, CENP-A and Shugoshins, and explore whether they are interrelated. In this chapter, the latter aim will be discussed based on the literature and data collected in this work. Furthermore, both experimental and theoretical approaches were adopted for this work, providing an opportunity to reflect on the use of the two methodologies in centromere biology research. 

\subsection{The potential link between the spatial pattern of CENP-A nucleosomes and tension sensing}

CENP-A nucleosomes display a non-random distribution at the centromere, where they form distinct islands interspersed by canonical H3 nucleosomes \citep{Blower2002ConservedHumans, Dunleavy2011H3.3Phase., Kyriacou2018}. On the other hand, the inner/ peri-centromeric localisation of Shugoshins is altered in response to tension, which is hypothesised to be important for tension sensing \citep{Huang2007, Liu2013, Asai2020, Lee2008, Gomez2007, Eshleman2014, Nerusheva2014, Paldi2020ConvergentPericentromeres, Clarke2005, Kawashima2007}. The idea that these two phenomena are relevant lies in the speculations that \textbf{a)} the spatial pattern of CENP-A nucleosomes is required to localise Shugoshins properly to sense tension and that \textbf{b)} it might enable the centromeric chromatin mechanical properties necessary for tension sensing. 

The first speculation was based on evidence from budding yeast. In this organism, the kinetochore facilitates the targeted loading of cohesin \citep{Hinshaw2015StructuralLoading, Hinshaw2017TheComplex, Fernius2009EstablishmentCsm3, Fernius2013Cohesin-DependentEstablishment, Natsume2013KinetochoresRecruitment}, which is able to shape peri-centromeric chromatin conformation \citep{Paldi2020ConvergentPericentromeres}. Altering the conformation caused a change in the ChIP pattern of Sgo1 \citep{Paldi2020ConvergentPericentromeres}. It is unclear whether this is due to the direct binding of Sgo1 to cohesin as in humans \citep{Liu2013, Hara2014, Garcia-Nieto2023StructuralProtection} or that the spatial distribution of H2A-S121 phosphorylation by Bub1 is dependent on the organisation of peri-centromeric chromatin. Either way, the chromatinic distribution of Sgo1 is linked to the location of the CENP-A nucleosome. Generalising this speculation to higher eukaryotes, which have more CENP-A nucleosomes clustered into islands in 1D, one can deduce that the spatial pattern of CENP-A nucleosomes is important for the proper localisation of Sgo1 required for tension sensing. However, in this work, it was found that cohesin-depleted cells, examined by microscopy, were still able to localise Sgo1 to the peri-centromeric region, which suggested that, at least at the coarse-grain level, cohesin is not required for the localisation of Sgo1 (Figure~\ref{fig:sgo1metscc1}). More importantly, as in \cite{Indjeian2005a}, the cell cycle was arrested in this case, indicating an intact tension sensing upon cohesin depletion. Together, they argued against the speculation above, where cohesin connects the CENP-A pattern to Shugoshins localisation and tension sensing. Alternatively, the speculation could be subject to other experimental tests. For example, a natural deduction from the speculation is that Shugoshins should localise at the intervals between CENP-A islands on a 1D centromere in higher eukaryotes. A Sgo1 IF experiment using the extended chromatin technique \citep{Blower2002ConservedHumans, Dunleavy2011H3.3Phase., Kyriacou2018} shall be able to test the prediction. 

Centromeric chromatin is more condensed than canonical one because of CENP-C and CENP-N, CCAN components recruited by CENP-A nucleosomes \citep{Geiss2014, Panchenko2011, Zhou2022}. Its physical properties might further be affected by the SMC complexes cohesin and condensin enriched locally \citep{Verzijlbergen2014, Haase2012Bub1Dynamics, Paldi2020ConvergentPericentromeres}. It is possible that these unique properties are necessary for tension sensing, assuming that tension sensing happens at the centromeric chromatin. However, the exact location for tension sensing has been a long-standing  debate \citep{McVey2021AuroraSegregation}. Since both the kinetochore and centromeric chromatin undergo deformation upon tension \citep{Goshima2000EstablishingYeast, Roscioli2020}, inter- and intra-kinetochore tension has been individually proposed to be important. A final conclusion is hindered because each of them is supported by extensive experimental evidence. In this work, it was found that the localisation of the peri-centromeric protein Sgo1 follows the localisation of the kinetochore protein Bub1 in response to tension (Figure~\ref{fig:bub1aid} and ~\ref{fig:bub1sgo1}). This result implies that it is likely to be the intra-kinetochore rather than the inter-kinetochore tension that is sensed, arguing against the assumption of the second speculation. 

In summary, the experimental results of this work do not support the two speculations and therefore not in favour of the hypothesis that the spatial pattern of CENP-A nucleosomes and tension sensing are related. However, the results above could not provide a direct test of the hypothesis. The experiment for this purpose should be examining tension sensing in cells with controlled distribution of CENP-A nucleosomes. A possible setting would be using budding yeast strains with engineered centromeres which contain several copies of consecutive or interspersed CDEI-III sequences. 

\subsection{The use of experimental and theoretical methods in centromere biology research}

% Benefits of Mathematical Modeling
% Humans make models to make sense of the world—to organize information and provide predictability. In the biological sciences, models are commonly expressed using words or diagrams, which often suffer from a lack of preci- sion. Mathematics is the most precise language we have for describing the processes we think are happening and for comparing this with what we can observe. Even attempting to make a mathematical model helps to clarify ideas, define assumptions and identify missing information. A working mathemati- cal description is a potent tool, simplifying large amounts of information, providing precise predictions to aid further experimental tests, and allowing rapid exploration of the range of properties of the system to help understand design–function relationships. Importantly, a mathematical model also aids generalization of insights gained from one system to other systems.
% However, the rigour of a mathematical model is no guarantee that it is correct. Models will always be based on incomplete and often incorrect knowledge. They should be judged by their usefulness—their ability to explain current knowledge and to guide the attainment of new knowledge.
% Our Modeling Approach—Simplified Stochastic Simulations
% There are many kinds of mathematical models (many examples are provided in this book). Our approach has been to use computer-based stochastic sim- ulations, which are a relatively “mathematics-light” way to add or remove possible processes and which naturally provide noise. The battle between a parameter-averse physicist (“We don’t need to put that in”) and a parameter- addicted biologist (“But we know it’s not that simple”) results in a compro- mise where complex processes are represented in simplified or idealized ways. Our hope is that this simplification exposes rather than obscures the funda- mentals of the system. Details are important, of course, but fundamentals are more important.