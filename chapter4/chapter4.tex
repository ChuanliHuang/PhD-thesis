\chapter{Discussion}

\section{Final discussion}

Accurate chromosome segregation is a fundamental requirement for eukaryotic life, which relies on the specialised locus called the centromere. The spatial regulation of critical proteins is the key to the realisation of centromeric function. This work aimed to study the spatial organisation and dynamics of two vital centromeric proteins, CENP-A and Shugoshins, and explore whether they are interrelated. In this chapter, the latter aim will be discussed based on the literature and data collected in this work. Furthermore, both experimental and theoretical approaches were adopted for this work, providing an opportunity to reflect on the use of the two methodologies in centromere biology research. 

\subsection{The potential link between the spatial pattern of CENP-A nucleosomes and tension sensing}

CENP-A nucleosomes display a non-random distribution at the centromere, where they form distinct islands interspersed by canonical H3 nucleosomes \citep{Blower2002ConservedHumans, Dunleavy2011H3.3Phase., Kyriacou2018}. On the other hand, the inner/ peri-centromeric localisation of Shugoshins is altered in response to tension, which is hypothesised to be important for tension sensing \citep{Huang2007, Liu2013, Asai2020, Lee2008, Gomez2007, Eshleman2014, Nerusheva2014, Paldi2020ConvergentPericentromeres, Clarke2005, Kawashima2007}. The idea that these two phenomena are relevant lies in the speculations that \textbf{a)} the spatial pattern of CENP-A nucleosomes is required to localise Shugoshins properly to sense tension and that \textbf{b)} it might enable the centromeric chromatin mechanical properties necessary for tension sensing. 

The first speculation was based on evidence from budding yeast. In this organism, the kinetochore facilitates the targeted loading of cohesin \citep{Hinshaw2015StructuralLoading, Hinshaw2017TheComplex, Fernius2009EstablishmentCsm3, Fernius2013Cohesin-DependentEstablishment, Natsume2013KinetochoresRecruitment}, which is able to shape peri-centromeric chromatin conformation \citep{Paldi2020ConvergentPericentromeres}. Altering the conformation caused a change in the ChIP pattern of Sgo1 \citep{Paldi2020ConvergentPericentromeres}. It is unclear whether this is due to the direct binding of Sgo1 to cohesin as in humans \citep{Liu2013, Hara2014, Garcia-Nieto2023StructuralProtection} or that the spatial distribution of H2A-S121 phosphorylation by Bub1 is dependent on the organisation of peri-centromeric chromatin. Either way, the chromatinic distribution of Sgo1 is linked to the location of the CENP-A nucleosome. Generalising this speculation to higher eukaryotes, which have more CENP-A nucleosomes clustered into islands in 1D, one can deduce that the spatial pattern of CENP-A nucleosomes is important for the proper localisation of Sgo1 required for tension sensing. However, in this work, it was found that cohesin-depleted cells, examined by microscopy, were still able to localise Sgo1 to the peri-centromeric region, which suggested that, at least at the coarse-grain level, cohesin is not required for the localisation of Sgo1 (Figure~\ref{fig:sgo1metscc1}). More importantly, as in \cite{Indjeian2005a}, the cell cycle was arrested in this case, indicating an intact tension sensing upon cohesin depletion. Together, they argued against the speculation above, where cohesin connects the CENP-A pattern to Shugoshins localisation and tension sensing. Alternatively, the speculation could be subject to other direct experimental tests. For example, a natural deduction from the speculation is that Shugoshins should localise at the intervals between CENP-A islands on a 1D centromere in higher eukaryotes. A Sgo1 IF experiment using the extended chromatin technique \citep{Blower2002ConservedHumans, Dunleavy2011H3.3Phase., Kyriacou2018} or Sgo1 ChIP-seq in higher eukaryotes shall be able to test the prediction. 

Centromeric chromatin is more condensed than canonical one because of CENP-C and CENP-N, CCAN components recruited by CENP-A nucleosomes \citep{Geiss2014, Panchenko2011, Zhou2022}. Its physical properties might further be affected by the SMC complexes cohesin and condensin enriched locally \citep{Verzijlbergen2014, Haase2012Bub1Dynamics, Paldi2020ConvergentPericentromeres}. It is possible that these unique properties are necessary for tension sensing, assuming that tension sensing happens at the centromeric chromatin. However, the exact location for tension sensing has been a long-standing  debate \citep{McVey2021AuroraSegregation}. Since both the kinetochore and centromeric chromatin undergo deformation upon tension \citep{Goshima2000EstablishingYeast, Roscioli2020}, inter- and intra-kinetochore tension has been individually proposed to be important. A final conclusion is hindered because each of them is supported by extensive experimental evidence. In this work, it was found that the localisation of the peri-centromeric protein Sgo1 follows the localisation of the kinetochore protein Bub1 in response to tension (Figure~\ref{fig:bub1aid} and ~\ref{fig:bub1sgo1}). This result implies that it is likely to be the intra-kinetochore rather than the inter-kinetochore tension that is sensed, arguing against the assumption of the second speculation. 

In summary, the experimental results of this work do not support the two speculations and therefore not in favour of the hypothesis that the spatial pattern of CENP-A nucleosomes and tension sensing are related. However, the results above could not provide a direct test of the hypothesis. The experiment for this purpose should be examining tension sensing in cells with controlled distribution of CENP-A nucleosomes. A possible setting would be using budding yeast strains with engineered centromeres which contain several copies of consecutive or interspersed CDEI-III sequences. 

\subsection{A reflect on the use of experimental and theoretical approaches in centromere biology research}

Empiricism and rationalism are the two dominant schools of philosophy for the source of knowledge in epistemology \citep{Solomon2006ThePhilosophy}. The former refers to the idea that all knowledge comes from experience while the latter emphasises the sole reliance of knowledge acquisition on human reason. In science, the two philosophies were developed into the method focusing on the collection and processing of data, the experimental approach, and the method by which the relationships between abstract entities are deduced, usually in the language of mathematics for its preciseness, the theoretical approach. 
Notably, both approaches are capable of forming models, the conceptual representations of the system of interest. It is only the types of models generated that are different. Experimental approaches usually give rise to qualitative models, where abstract concepts are described by words or diagrams, and statistical models, where quantitative behaviours of the system are described by mathematical expressions of best fit without considering the physical realities. Conversely, the models mathematically describing the system based on actual physics, the mechanistic models, can only be derived by theoretical approaches. 

The advantage of experimental approaches resides in the closeness to facts. But the difficulty of interpreting results increases with the complexity of the system studied. On the contrary, the strength and weakness of theoretical approaches are the opposite. It facilitates understanding the basic principles underneath the system by providing a rigorous framework. However, this is at risk of being irrelevant to the actual system if false assumptions are used, because the guarantee of true conclusions by deduction requires not only valid reasoning but also correct premises. The complementary nature of experimental and theoretical approaches results in their combination being considered the standard scientific methodology in various disciplines \citep{Platt1964StrongInference}. Nevertheless, in biology, there has been an imbalance between experiment and theory \citep{Fidelman1985TheModeling}. As an estimate, on 25th May 2023, 23,205 results were found when 'centromere' was typed in PubMed search box whereas the number is merely 172 for 'centromere + mathematical'. \cite{Fidelman1985TheModeling} listed unfavourable outcomes caused by this imbalance that delays the progress of the field, which might still be relevant today. First, they suggested that the imbalance had led to a growth of data with weak interpretation. Consistently, it was recently reported that there has been no increase in the number of important conclusions despite the explosion of literature \citep{Park2023PapersTime, Nurse2021BiologyData}. Naturally, this would place an increasingly heavier burden on researchers to keep themselves updated. There could even be confusion in the field of what has been established, resulting in repetitive work testing proven hypotheses. Second, the dominance of experimental approaches had weakened experimentalists' need to understand and interpret theoretical work. As a result, unlike in physics or chemistry, experimentalists in biology usually do not possess the quantitative background required, causing a communication gap between them and theoreticians. The authors exemplified this point by describing a typical meeting organised with the aim to reduce the gap. After presentations given by theoreticians, an experimentalist would open the presentation by reassuring his or her colleagues that it would be free of models and theories. Third, it was proposed that the imbalance in weight had brought about an imbalance in importance. Theoreticians often rely on experimentalists to conduct research rather than working independently due to the culture of the field, which is materialised by the preference of funding bodies and publishers. This makes experimentalists the core investigators whereas leaves the theoreticians a more service role, which would discourage young scientists from pursuing theoretical biology and further increase the imbalance of research approaches. 

I can relate my experience of having worked in experimental and theoretical groups to the problems \cite{Fidelman1985TheModeling} raised above. Regarding the first point, it is true that, while conducting the experimental project, I frequently had difficulty distinguishing agreed conclusions from proposed speculations in the field and therefore interpreting the experimental data collected. To the second point, although experimental and theoretical approaches have their own pros and cons, an unpleasant phenomenon I encountered is that experimentalists and theoreticians tend to focus on the shortness rather than the strength of each other's methodology, which causes tension, or even dislike, between the two groups of researchers. In terms of the third point, my observation is consistent with what was described above. Theoreticians are often asked to model data or ideas from experimentalists but it is rare that experimentalists conduct experiments based on the predictions of models theoreticians built. From a personal point of view, at least in the field of centromere biology, the imbalance between experimental and theoretical approaches and the outcomes it caused still persist. It would be of great benefit if more theoretical effort was included in future research. Not only because it might mitigate the problems mentioned, but also due to lower cost, shorter project cycle and fewer ethical issues it could provide. 

Although firmly support the idea that experimental biologists should be more aware of quantitative knowledge to understand theoretical work, I do not recommend individual scientists to conduct research using both approaches simultaneously. Each of the methods has a long history of development and is therefore highly specialised. Mastering any of them requires years of training. Keeping oneself updated about new development in the field also needs regular effort. Furthermore, the technical skills for each methodology are almost mutually exclusive. The experimental approach asks for handling cells and using scientific equipment whereas the theoretical approach requests formulating mathematical equations and programming. Practically, the time taken for one to be able to apply both approaches in actual research would be enormous. Apart from the steep learning curve, a more fundamental problem I faced with adopting both methods was the different mindsets required. From my personal experience, experimental work involves a lot of planning and execution, therefore demanding a mindset of discipline and carefulness. However, theoretical work emphasises more thinking, which requires a mindset of imagination and open-mindedness. Switching between the two mindsets was challenging. I often found the experiments more error-prone after completing some theoretical work. Similarly, theoretical work could suffer from low productivity if an experiment was executed before. Hence, instead of attempting to use both approaches simultaneously by one researcher, a probably better mode would be the collaboration between experimentalists and theoreticians, as what usually occurs in physics and chemistry. 





% Big future directions of how centromeres and pericentromeres can be understood better by the types of approaches you had done in your PhD.
%   -theory and experiment (pros and cons)
%   -imbalance of theory and experiment in  biology and its damaging results (More theory is needed)
%     -growth of literature full of data with weak interpretation
%     -communication gap between experimentalists and theoreticians
%     -work ethic: overly emphasise data collection over developing theory, rather than doing the right experiment, keep doing experiments blindly 
%   -Incompatibility of theoretical and experimental work at the same time

% Benefits of Mathematical Modeling
% Humans make models to make sense of the world—to organize information and provide predictability. In the biological sciences, models are commonly expressed using words or diagrams, which often suffer from a lack of preci- sion. Mathematics is the most precise language we have for describing the processes we think are happening and for comparing this with what we can observe. Even attempting to make a mathematical model helps to clarify ideas, define assumptions and identify missing information. A working mathemati- cal description is a potent tool, simplifying large amounts of information, providing precise predictions to aid further experimental tests, and allowing rapid exploration of the range of properties of the system to help understand design–function relationships. Importantly, a mathematical model also aids generalization of insights gained from one system to other systems.
% However, the rigour of a mathematical model is no guarantee that it is correct. Models will always be based on incomplete and often incorrect knowledge. They should be judged by their usefulness—their ability to explain current knowledge and to guide the attainment of new knowledge.
% Our Modeling Approach—Simplified Stochastic Simulations
% There are many kinds of mathematical models (many examples are provided in this book). Our approach has been to use computer-based stochastic sim- ulations, which are a relatively “mathematics-light” way to add or remove possible processes and which naturally provide noise. The battle between a parameter-averse physicist (“We don’t need to put that in”) and a parameter- addicted biologist (“But we know it’s not that simple”) results in a compro- mise where complex processes are represented in simplified or idealized ways. Our hope is that this simplification exposes rather than obscures the funda- mentals of the system. Details are important, of course, but fundamentals are more important.

% logic does not guarantee the truth unless the assumption is correct 

% PubMed centromere + mathetmatical 172 results; centromere 23,205

% different skills
% highly specialised
% Being able to conduct research work requires much more time investment than being able to understand the methodology 

% different mindset (perhaps more fundamental)

% be specialised in one and then collaborate

% It is perhaps why experimental and theoretical work are usually separated in physics and chemistry. 