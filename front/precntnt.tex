%%
%% precntnt.tex - LaTeX2e thesis class
%%
%% Copyright (C) 2010-2021 Mathew Topper <damm_horse@yahoo.co.uk>
%%
%%
%%   ABOUT
%%
%% This is frontmatter prior to the contents for a Latex2e template which
%% corresponds to the regulations regarding layout of a thesis submitted
%% within the University of Edinburgh. 
%%
%% INPUT THIS FILE USING THE /makefrontmatter{} COMMAND OR THE FORMATTING
%% WON'T WORK PROPERLY

%%%% DEDICATION

% \dedication{%
% \begin{normalsize}To my dearly departed cat,\end{normalsize}\\[0.2cm]%
% Roxy%
% }
\dedication{%
\begin{normalsize}To my family and friends\end{normalsize}
}

%%%% ABSTRACT
\abstract{Eukaryotes package their genomes into chromosomes, which must be replicated and segregated equally during each cell division. Centromere is the specialised locus for faithful chromosome segregation. The spatial organisation and dynamics of proteins are common practices that biological systems employ to dramatically increase their complexity given a certain amount of genetic information, which enables the remarkable richness of behaviours required to meet the tough challenges they face. The same applies to the centromere, where its function of ensuring accurate chromosome segregation is partially, if not largely, realised by regulating the spatial organisation of critical proteins. Mechanistic explanations for many of these fascinating processes are still lacking. In this work, I studied the spatial organisation and dynamics of two key centromeric proteins, CENP-A and Shugoshin, attempting to uncover their respective underlying mechanisms. 

CENP-A
bg
methods
results

Shuogshin
}


%%%% 摘要
\begin{CJK*}{UTF8}{gbsn}
\chapter{\begin{CJK*}{UTF8}{gbsn}摘要\end{CJK*}}
中文测试
\end{CJK*}

%%%% LAY SUMMARY
\summary{See the web page
\href{http://www.dcc.ac.uk/resources/how-guides/write-lay-summary}{"How to Write a Lay Summary"}
for a guide.}

%%%% ACKNOWLEDGEMENTS

\acknowledgements{%
I'd like to thank Danger Mouse, for being so awesome.}

%%%% DECLARATION

%% Use a custom declaration

% \declaration{I did it.}

%% Use the standard regulation declaration. Enter your
%% name for the signature line.

\standarddeclaration{Chuanli Huang}
