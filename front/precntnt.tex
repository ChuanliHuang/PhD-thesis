%%
%% precntnt.tex - LaTeX2e thesis class
%%
%% Copyright (C) 2010-2021 Mathew Topper <damm_horse@yahoo.co.uk>
%%
%%
%%   ABOUT
%%
%% This is frontmatter prior to the contents for a Latex2e template which
%% corresponds to the regulations regarding layout of a thesis submitted
%% within the University of Edinburgh. 
%%
%% INPUT THIS FILE USING THE /makefrontmatter{} COMMAND OR THE FORMATTING
%% WON'T WORK PROPERLY

%%%% DEDICATION

% \dedication{%
% \begin{normalsize}To my dearly departed cat,\end{normalsize}\\[0.2cm]%
% Roxy%
% }
\dedication{%
\begin{normalsize}To my family and friends\end{normalsize}
}

%%%% ABSTRACT
\abstract{Eukaryotes package their genomes into chromosomes, which must be replicated and segregated equally during each cell division. Centromere is the specialised locus for faithful chromosome segregation. The spatio-temporal organisation of proteins is a common strategy that biological systems adopt to increase their complexity given a certain amount of genetic information, which provides the degrees of freedom required to meet the diverse challenges they face. The same applies to the centromere, where its function of ensuring accurate chromosome segregation is partially, if not largely, realised by regulating the spatial organisation of critical proteins. Mechanistic understanding of many of these fascinating processes is still lacking. In this work, I studied the spatio-temporal organisation and dynamics of two key centromeric proteins, CENP-A and Shugoshin, attempting to uncover their respective underlying mechanisms. 

CENP-A is the histone H3 variant whose presence epigenetically defines centromere identity. The establishment and maintenance of CENP-A nucleosomes on chromatin is one of the key questions in centromere biology. As with other epigenetics marks, CENP-A nucleosomes are diluted because of DNA replication and have to be replenished in each cell cycle. A self-templating local deposition mechanism has emerged from experimental evidence but it fails to explain several characteristics of CENP-A spatial organisation. To address the problem, I built a theoretical model for the dynamic dilution and replenishment of CENP-A nucleosomes. The classic epigenetics model 'beads-on-string' was modified for CENP-A. It was found that local auto-amplification alone is insufficient to recapitulate the experimental observations, suggesting the existence of other mechanisms. Further analysis indicated that cooperativity and arbitrary number control are unlikely to be one of them. The combination of spontaneous conversion and threshold-based stabilisation gave the best mimicry of the target characteristics. These results could provide directions for future experiments aiming to understand the establishment and maintenance of CENP-A nucleosomes. 

Shugoshin is a peri-centromeric adaptor protein crucial for accurate chromosome segregation. The localisation of Shugoshin is responsive to tension, a mechanical force unique to bi-orientated sister chromatids, in various species. Despite the functional characterisation of this tension-dependent re-localisation, the underlying mechanisms are still elusive. I studied the question using experimental approaches in the model organism budding yeast \textit{Saccharomyces cerevisiae}. There is only one Shugoshin variant named Sgo1 in budding yeast. The SAC kinase Bub1 and its substrate H2A-S121 have been shown to be important for the peri-centromeric localisation of Sgo1. In this study, I found that, upon the establishment of tension, the kinetochore localisation of Bub1 is largely reduced and that H2A-S121 phosphorylation re-distributes from the centromere to chromosome arms, suggesting a sequential re-localisation of Bub1, H2A-S121 phosphorylation and Sgo1. Consistently, the removal of Bub1 promptly abolished H2A-S121 phosphorylation and Sgo1 peri-centromere localisation. This model indicates the existence of at least one phosphatase antagonising the activity of Bub1. Although PP1 is required for the re-localisation of Sgo1, the requirement was shown to be because of its role in regulating Bub1 localisation. The Sgo1 interactor PP2A-Rts1 was also found to be unimportant in this process, leaving the phosphatase(s) unidentified. Since Sgo1 has been proposed to promote chromosome condensation, our sequential re-distribution model predicts a more condensed chromosome status upon tension. Re-analysis of published Hi-C data confirmed the prediction, further supporting the model. Taken together, this work suggested a qualitative model for the tension-dependent re-localisation of Sgo1, provided insights into the mechanism of tension sensing and implied a potential link between SAC and chromosome condensation. 
}

%%%% 摘要
\begin{CJK*}{UTF8}{gkai}
\chapter{\begin{CJK*}{UTF8}{gkai}摘要\end{CJK*}}
真核生物将其基因组包装成为染色体。在每次细胞分裂过程中,染色体必须被复制和均等分配到子细胞,而着丝粒是保证染色体正确分离的专门位点。蛋白质的时空组织是生物系统在遗传信息量一定的情况下增加复杂性的常见策略,这为其应对所面临的繁杂挑战提供了所需的自由度。同样,着丝粒通过调控关键蛋白质的空间组织来实现染色体的准确分离,但其中许多过程仍然欠缺机制性理解。本课题研究了两个关键的着丝粒蛋白CENP-A和Shugoshin的时空组织和动态,尝试揭示它们各自可能的作用机制。

CENP-A是一种组蛋白H3变体,其存在以表观遗传的形式定义了着丝粒的身份。CENP-A核小体如何在染色质上建立和维持是着丝粒研究的关键问题之一。与其他表观遗传标记一样,在每个细胞周期中CENP-A核小体的数量会因DNA复制而被稀释,所以需要重新补充。使用实验方法的研究提出了一种自模板的局部投放机制,但是其无法解释CENP-A空间组织上的若干重要特征。对此,基于经典的表观遗传理论模型,我构建了一个CENP-A核小体动态稀释和补充的理论模型。研究结果发现,仅靠局部自增强无法复现实验观察结果,这暗示了其他机制的存在。进一步分析表明,协同作用和数量控制不太可能是这些机制之一。而自发转换与基于阈值的稳定化的结合能够最大程度的模拟出目标特征。这项工作可以为未来旨在理解CENP-A核小体的建立和维持的实验提供方向。

Shugoshin是一个近着丝粒适配蛋白,对染色体的准确分离至关重要。其定位能够响应仅存在于双向定向的姐妹染色单体中的张力。尽管相关研究已经报道过这种张力响应再定位的生物学功能,但其潜在机制仍然不明。对此,我使用模式生物出芽酵母对此进行了实验研究。在出芽酵母中,只有一个Shugoshin变体——Sgo1。先前研究报道了SAC激酶Bub1及其底物H2A-S121对Sgo1的近着丝粒定位的重要作用。本研究中,我发现,一旦张力产生,Bub1在动粒的分布大幅减少,且H2A-S121磷酸化从近着丝粒扩散至染色体臂,这隐射了Bub1、H2A-S121磷酸化和Sgo1三者的按顺序再定位。与此一致,Bub1的实时敲除快速清除了H2A-S121磷酸化和Sgo1近着丝粒定位。这个模型暗示至少存在一个与Bub1活性相对抗的磷酸酶。尽管实验结果显示PP1对于Sgo1的再定位是必需的,但后续研究发现这只是它调控Bub1定位的联动效应。Sgo1的互作蛋白PP2A-Rts1也被证明在这个过程中并不重要。所以,该磷酸酶的身份仍然未知。由于Sgo1被认为可以促进染色体凝集,根据我们的按顺序再定位模型的预测,在张力存在的情况下染色体构型会更加紧密。对已发表的Hi-C数据的重新分析证实了这一预测,进一步支持了该模型。因此,这项工作提出了一个Sgo1张力响应再定位的定性模型,加深了对张力感应机制的理解,发现了SAC与染色体凝集之间的潜在联系。
\end{CJK*}

%%%% LAY SUMMARY
\summary{The growth, healing and reproduction of all living organisms rely on cell division. Delivering an intact genome to the daughter cell challenges every dividing cell. Eukaryotes package their genomes into chromosomes, which must be replicated and segregated equally during each cell division. Severe diseases, such as cancer, genetic disorder and miscarriage, could arise if chromosome segregation is compromised. Centromere is the specialised locus for chromosome segregation, where the chromosome is physically attached to the spindle. First described as primary constriction, it is visualised as the intersection point of the iconic X-shaped mitotic chromosome pictured in biology textbooks. One can imagine that impaired centromere function will lead to devastating errors in chromosome segregation. 

The spatio-temporal organisation of proteins is a common strategy that biological systems adopt to increase their complexity given a certain amount of genetic information, which provides the degrees of freedom required to meet the diverse challenges they face. The same applies to the centromere, where its function of ensuring accurate chromosome segregation is partially, if not largely, realised by regulating the spatial organisation of critical proteins. Mechanistic understanding of many of these fascinating processes is still lacking. In this work, I studied the spatio-temporal organisation and dynamics of selected key centromeric proteins, which provides insights into centromere biology and could potentially benefit future medical research aiming to cure centromere-related diseases. 
}

%%%% ACKNOWLEDGEMENTS

\acknowledgements{
\begin{CJK*}{UTF8}{gkai}
I am grateful to my principal supervisor \textbf{Prof Andrew Goryachev} for giving me the opportunity to conduct this PhD, from which I learnt so much. His guidance and mentoring are beneficial for and beyond this PhD. 

I am equally grateful to my secondary supervisor \textbf{Prof Adele Marston} for allowing me to carry out research in her lab. Her support and encouragement are indispensable for this PhD. 

I would like to thank \textbf{Prof Patrick Heun} for being the chair of my thesis committee. His advice on scientific methodology is extremely helpful. 

I very much appreciate the people who provided me with technical support: \textbf{Dr David Kelly} and \textbf{Dr Toni Mchugh} from COIL for microscopy; \textbf{Dr Christos Spanos} from Proteomics Facility for proteomics; \textbf{Daniel Robertson} from Bioinformatics Core Facility for bioinformatics; \textbf{Dr Ivan Erofeev} from Goryachev lab for computer simulation; \textbf{Dr Bessie Su} (宿雪), \textbf{Dr Lori Koch}, \textbf{Dr Weronika Borek}, \textbf{Dr Lucia Massari}, \textbf{Dr Hollie Rowland} and \textbf{Melanie Lim} (林芷晴) from Marston lab for various wet lab techniques. 

I also appreciate my friends in or not in Edinburgh for providing me with life advice and mental support: \textbf{Dr Liangcui Chu} (初良萃), \textbf{Dr Sizhou Wu} (吴思洲), \textbf{Dr Yu Huo} (霍雨), \textbf{Haidao Zhang} (张海岛), \textbf{Xiang Wan} (万向), \textbf{Haonan Liu} (刘浩男), \textbf{Wei Li} (李伟), \textbf{Ziyu Zhao} (赵子瑜), \textbf{Renze Gao} (高仁泽), \textbf{Shaojie Ma} (马绍杰) and so on. 

Finally, I would like to express my sincere thanks to my family for everything they have done for me: \textbf{Xiang Zheng} (郑相), \textbf{Qinying Lu} (陆琴英), \textbf{Xi Huang} (黄熙), \textbf{Jingmei Lu} (陆\includegraphics[scale=0.06, trim=0 1cm 0 0]{front/pictures/jing.png}梅) and \textbf{Danlei Cheng} (成丹蕾). 
\end{CJK*}
}

%%%% DECLARATION

%% Use a custom declaration

% \declaration{I did it.}

%% Use the standard regulation declaration. Enter your
%% name for the signature line.

\standarddeclaration{Chuanli Huang}
