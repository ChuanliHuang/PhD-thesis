%%
%% precntnt.tex - LaTeX2e thesis class
%%
%% Copyright (C) 2010-2021 Mathew Topper <damm_horse@yahoo.co.uk>
%%
%%
%%   ABOUT
%%
%% This is frontmatter prior to the contents for a Latex2e template which
%% corresponds to the regulations regarding layout of a thesis submitted
%% within the University of Edinburgh. 
%%
%% INPUT THIS FILE USING THE /makefrontmatter{} COMMAND OR THE FORMATTING
%% WON'T WORK PROPERLY

%%%% DEDICATION

% \dedication{%
% \begin{normalsize}To my dearly departed cat,\end{normalsize}\\[0.2cm]%
% Roxy%
% }
\dedication{%
\begin{normalsize}To my family and friends\end{normalsize}
}

%%%% ABSTRACT
\abstract{Eukaryotes package their genomes into chromosomes, which must be replicated and segregated equally during each cell division. Centromere is the specialised locus for faithful chromosome segregation. The spatial organisation and dynamics of proteins are common practices that biological systems employ to increase their complexity, enabling the richness of behaviours required to meet the challenges they face. The same applies to the centromere, where its function of ensuring accurate chromosome segregation is partially, if not largely, realised by regulating the spatial organisation of critical proteins. Mechanistic explanations for many of these fascinating processes are still lacking. In this work, I studied the spatial organisation and dynamics of two key centromeric proteins, CENP-A and Shugoshin, attempting to uncover their respective underlying mechanisms. 

CENP-A is the histone H3 variant whose presence epigenetically defines centromere identity. The establishment and maintenance of CENP-A nucleosomes on chromatin is one of the key questions in centromere biology. As with other epigenetics marks, CENP-A nucleosomes are diluted because of DNA replication and have to be replenished in each cell cycle. A self-templating local deposition mechanism has emerged from experimental evidence but it fails to explain several characteristics of CENP-A spatial organisation. To address the problem, I built a theoretical model for the dynamic dilution and replenishment of CENP-A nucleosomes. The classic epigenetics model 'beads-on-string' was modified for CENP-A. It was found that local auto-amplification alone is insufficient to recapitulate the experimental observations, suggesting the existence of other mechanisms. Further analysis indicated that cooperativity and arbitrary number control are unlikely to be one of them. The combination of spontaneous conversion and threshold-based stabilisation gave the best mimicry of the target characteristics. These results could provide directions for future experiments aiming to understand the establishment and maintenance of CENP-A nucleosomes. 

Shugoshin is a peri-centromeric adaptor protein crucial for accurate chromosome segregation. The localisation of Shugoshin is responsive to tension, a mechanical force unique to bi-orientated sister chromatids, in various species. Despite the functional characterisation of this tension-dependent re-localisation, the molecular mechanisms underneath are still elusive. I studied the question using experimental approaches in the model organism budding yeast \textit{Saccharomyces cerevisiae}. There is only one Shugoshin variant, Sgo1, in this organism. The SAC kinase Bub1 and its substrate H2A-S121 have been shown to be important for the peri-centromeric localisation of Sgo1. In this study, I found that, upon the establishment of tension, the kinetochore localisation of Bub1 is largely reduced and that H2A-S121 phosphorylation re-distributes from the centromere to chromosome arms, suggesting a sequential re-localisation of Bub1, H2A-S121 phosphorylation and Sgo1. Consistently, the removal of Bub1 promptly abolished H2A-S121 phosphorylation and Sgo1 peri-centromere localisation. This model indicates the existence of at least one phosphatase antagonising the activity of Bub1. Although PP1 is required for the re-localisation of Sgo1, the requirement was shown to be because of its role in regulating Bub1 localisation. The Sgo1 interactor PP2A-Rts1 was also found to be unimportant in this process, leaving the phosphatase(s) unidentified. Since Sgo1 has been proposed to promote chromosome condensation, our sequential re-distribution model predicts a more condensed chromosome status upon tension. Re-analysis of published Hi-C data confirmed the prediction, further supporting the model. Taken together, this work suggested a qualitative model for the tension-dependent re-localisation of Sgo1, provided insights into the mechanism of tension sensing and implied a potential link between SAC and chromosome condensation. 
}


%%%% 摘要
\begin{CJK*}{UTF8}{gbsn}
\chapter{\begin{CJK*}{UTF8}{gbsn}摘要\end{CJK*}}
中文测试
\end{CJK*}

%%%% LAY SUMMARY
\summary{See the web page
\href{http://www.dcc.ac.uk/resources/how-guides/write-lay-summary}{"How to Write a Lay Summary"}
for a guide.}

%%%% ACKNOWLEDGEMENTS

\acknowledgements{%
I'd like to thank Danger Mouse, for being so awesome.
% andrew
% adele
% ivan 
% bessie lori wera lucia hollie aparna ola
% spanos
% kelly toni
% dan
% wei liangcui xiang haonan hana sizhou haidao yu
% melanie 
% francois miles
% ziyu renze yizhou yakun si
% shaojie 
% lu li zhengyi huang kaikai huang
% fei
% parents grandparents







}

%%%% DECLARATION

%% Use a custom declaration

% \declaration{I did it.}

%% Use the standard regulation declaration. Enter your
%% name for the signature line.

\standarddeclaration{Chuanli Huang}
