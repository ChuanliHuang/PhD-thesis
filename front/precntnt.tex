%%
%% precntnt.tex - LaTeX2e thesis class
%%
%% Copyright (C) 2010-2021 Mathew Topper <damm_horse@yahoo.co.uk>
%%
%%
%%   ABOUT
%%
%% This is frontmatter prior to the contents for a Latex2e template which
%% corresponds to the regulations regarding layout of a thesis submitted
%% within the University of Edinburgh. 
%%
%% INPUT THIS FILE USING THE /makefrontmatter{} COMMAND OR THE FORMATTING
%% WON'T WORK PROPERLY

%%%% DEDICATION

% \dedication{%
% \begin{normalsize}To my dearly departed cat,\end{normalsize}\\[0.2cm]%
% Roxy%
% }
\dedication{%
\begin{normalsize}To my family and friends\end{normalsize}
}

%%%% ABSTRACT
\abstract{Eukaryotes package their genomes into chromosomes, which must be replicated and segregated equally during each cell division. Centromere is the specialised locus for faithful chromosome segregation. The spatio-temporal organisation of proteins is a common strategy that biological systems adopt to increase their complexity given a certain amount of genetic information, which provides the degrees of freedom required to meet the diverse challenges they face. The same applies to the centromere, where its function of ensuring accurate chromosome segregation is partially, if not largely, realised by regulating the spatial organisation of critical proteins. Mechanistic understanding of many of these fascinating processes is still lacking. In this work, I studied the spatio-temporal organisation and dynamics of two key centromeric proteins, CENP-A and Shugoshin, attempting to uncover their respective underlying mechanisms. 

CENP-A is the histone H3 variant whose presence epigenetically defines centromere identity. The establishment and maintenance of CENP-A nucleosomes on chromatin is one of the key questions in centromere biology. As with other epigenetics marks, CENP-A nucleosomes are diluted because of DNA replication and have to be replenished in each cell cycle. A self-templating local deposition mechanism has emerged from experimental evidence but it fails to explain several characteristics of CENP-A spatial organisation. To address the problem, I built a theoretical model for the dynamic dilution and replenishment of CENP-A nucleosomes. The classic epigenetics model 'beads-on-string' was modified for CENP-A. It was found that local auto-amplification alone is insufficient to recapitulate the experimental observations, suggesting the existence of other mechanisms. Further analysis indicated that cooperativity and arbitrary number control are unlikely to be one of them. The combination of spontaneous conversion and threshold-based stabilisation gave the best mimicry of the target characteristics. These results could provide directions for future experiments aiming to understand the establishment and maintenance of CENP-A nucleosomes. 

Shugoshin is a peri-centromeric adaptor protein crucial for accurate chromosome segregation. The localisation of Shugoshin is responsive to tension, a mechanical force unique to bi-orientated sister chromatids, in various species. Despite the functional characterisation of this tension-dependent re-localisation, the underlying mechanisms are still elusive. I studied the question using experimental approaches in the model organism budding yeast \textit{Saccharomyces cerevisiae}. There is only one Shugoshin variant named Sgo1 in budding yeast. The SAC kinase Bub1 and its substrate H2A-S121 have been shown to be important for the peri-centromeric localisation of Sgo1. In this study, I found that, upon the establishment of tension, the kinetochore localisation of Bub1 is largely reduced and that H2A-S121 phosphorylation re-distributes from the centromere to chromosome arms, suggesting a sequential re-localisation of Bub1, H2A-S121 phosphorylation and Sgo1. Consistently, the removal of Bub1 promptly abolished H2A-S121 phosphorylation and Sgo1 peri-centromere localisation. This model indicates the existence of at least one phosphatase antagonising the activity of Bub1. Although PP1 is required for the re-localisation of Sgo1, the requirement was shown to be because of its role in regulating Bub1 localisation. The Sgo1 interactor PP2A-Rts1 was also found to be unimportant in this process, leaving the phosphatase(s) unidentified. Since Sgo1 has been proposed to promote chromosome condensation, our sequential re-distribution model predicts a more condensed chromosome status upon tension. Re-analysis of published Hi-C data confirmed the prediction, further supporting the model. Taken together, this work suggested a qualitative model for the tension-dependent re-localisation of Sgo1, provided insights into the mechanism of tension sensing and implied a potential link between SAC and chromosome condensation. 
}


%%%% 摘要
\begin{CJK*}{UTF8}{gbsn}
\chapter{\begin{CJK*}{UTF8}{gbsn}摘要\end{CJK*}}
真核生物将其基因组包装成为染色体。在每次细胞分裂过程中,染色体必须被复制和平均分配,而着丝粒是保证染色体正确分离的特种位点。蛋白质的时空组织是生物系统在遗传信息一定的情况下增加复杂性的常见策略,为其应对所面临的繁杂挑战提供了所需的自由度。这也同样适用于着丝粒,它通过调控关键蛋白质的空间组织来实现染色体的准确分离。许多这些令人着迷的过程仍然欠缺机制性理解。在此项工作中,我研究了两个关键的着丝粒蛋白质——CENP-A和Shugoshin的时空组织和动态,尝试揭示它们各自的潜在机制。

CENP-A是一种组蛋白H3变体,其存在以表观遗传的形式定义了着丝粒的身份。CENP-A核小体如何在染色质上建立和维持是着丝粒研究的关键问题之一。与其他表观遗传标记一样,CENP-A核小体在每个细胞周期中会因DNA复制而稀释并且需要重新补充。使用实验方法的研究支持一种自模板的局部投放机制,但其无法解释CENP-A空间组织上的若干重要特征。为了解决这个问题,我构建了一个CENP-A核小体动态稀释和补充的理论模型。经典的表观遗传理论模型“绳珠模型”被修改应用于CENP-A系统。研究发现,仅靠局部自增强是不足以复现实验观察结果的,显示了其他机制的存在。进一步分析表明,协同作用和数量控制不太可能是其中之一。而自发转换与基于阈值的稳定化的结合能够给出目标特征最好的模拟。这些结果可以为未来旨在理解CENP-A核小体的建立和维持的实验提供方向。

Shugoshin是一个对准确染色体分离至关重要的近着丝粒适配蛋白,其定位对于张力具有响应性。张力在这里指仅存在于双向定向的姐妹染色单体中的机械力。尽管这种响应张力的再定位的生物学功能已被描述,但其潜在机制仍然不明。我使用出芽酵母\textit{Saccharomyces cerevisiae}这一模型生物对此进行了实验研究。在出芽酵母中,只有一个名为Sgo1的Shugoshin变体。过往文献显示了SAC激酶Bub1及其底物H2A-S121对Sgo1的近着丝粒定位的重要作用。在这项研究中,我发现,一旦张力建立,Bub1在动粒的定位大幅减少,且H2A-S121磷酸化从着丝粒重新分布至染色体臂,这隐射了Bub1、H2A-S121磷酸化和Sgo1的按顺序再定位。与此一致,去除Bub1立即消除了H2A-S121磷酸化和Sgo1近着丝粒定位。这个模型暗示至少存在一个与Bub1活性相对抗的磷酸酶。尽管PP1对于Sgo1的重新定位是必需的,但实验结果显示这是由于它在调控Bub1定位方面的作用。Sgo1的互作蛋白PP2A-Rts1也被发现在这个过程中并不重要。所以,该磷酸酶的身份仍然未知。由于Sgo1被认为可以促进染色体凝集,我们的按顺序再定位模型预测在张力作用下染色体构型会更加紧密。对已发表的Hi-C数据的重新分析证实了这一预测,进一步支持了该模型。因此,这项工作提出了一个Sgo1响应张力的再定位的定性模型,提供了对张力感应机制的洞见,发现了SAC与染色体凝集之间的潜在联系。
\end{CJK*}

%%%% LAY SUMMARY
\summary{See the web page
\href{http://www.dcc.ac.uk/resources/how-guides/write-lay-summary}{"How to Write a Lay Summary"}
for a guide.}

%%%% ACKNOWLEDGEMENTS

\acknowledgements{%
I'd like to thank Danger Mouse, for being so awesome.
% andrew
% adele
% ivan 
% bessie lori wera lucia hollie aparna ola
% spanos
% kelly toni
% dan
% wei liangcui xiang haonan hana sizhou haidao yu
% melanie 
% francois miles
% ziyu renze yizhou yakun si
% shaojie 
% lu li zhengyi huang kaikai huang
% fei
% parents grandparents







}

%%%% DECLARATION

%% Use a custom declaration

% \declaration{I did it.}

%% Use the standard regulation declaration. Enter your
%% name for the signature line.

\standarddeclaration{Chuanli Huang}
